\documentclass[11pt]{article}
\usepackage{amsmath}
\usepackage{amsfonts}
\usepackage[left=1in,right=1in,top=1in,bottom=1in]{geometry}

%opening
\title{GMBE Derivation}
\author{Ryan M. Richard}

\newcommand{\card}[1]{\left| #1\right|}
\newcommand{\fxn}[1]{f\left(#1\right)}
\newcommand{\gxn}[1]{g\left(#1\right)}
\newcommand{\pset}[1]{\mathcal{P}\left(#1\right)}
\newcommand{\mcn}[2]{\mathcal{U}^{(#1)}\left(#2\right)}
\newcommand{\fn}[1]{f^{(#1)}}
\newcommand{\ints}[2]{\mathcal{I}^{(#1)}\left(#2\right)}
\newcommand{\iset}[1]{\mathcal{I}\left(#1\right)}
\newcommand{\setf}{\mathbb{F}}
\newcommand{\fmem}[2]{f^{\left(#1\right)}_{#2}}
\newcommand{\smem}[2]{s^{\left(#1\right)}_{#2}}
\newcommand{\setfn}[1]{\setf^{(#1)}}
\newcommand{\sets}{\mathbb{S}^{(N)}}
\newcommand{\setx}{\mathbb{X}}
\newcommand{\fxns}{\fxn{\pset{\sets}}}
\newcommand{\gapprox}[1]{g\left(#1\right)}

\begin{document}

\maketitle

\section{Notation}

Let:

\begin{itemize}
	\item $\sets$ denote a set of cardinality $N$
	\item $\pset{\sets}$ be the powerset of the set $\sets$.
	\item $\mcn{n}{\sets}$ is the subset of $\pset{\sets}$, whose elements are
          (unions of) $n$-length combinations of elements taken from $\sets$
    \item $\ints{n}{\sets}$ is similar to $\mcn{n}{\sets}$ except that instead
         of taking unions of $n$-length combinations one takes intersections.
    \item By analogy to powerset we define $\iset{\sets}$ to be the set: $\left\lbrace\ints{m}{\sets} \mid m\in[0,N]\right\rbrace$.
\end{itemize}

\section{Problem Descriptions}

Ultimately we have a function $f$, which maps the powerset of a set $\sets$ to a
scalar $\fxns$; however, the computational cost of
computing $\fxns$ scales exponentially with $N$. Instead we propose a series of
systematic approximations to $\fxns$.

In the first order approximation we form a family of sets $\setfn{m}$ which
contains $m$ subsets of $\sets$. The only stipulation on the members of
$\setfn{m}$ is that:
\begin{equation}
    \sets = \bigcup_{\fmem{1}{i}\in\setfn{m}} \fmem{1}{i}
\end{equation}
must be true. Of note we do not assume that the various $\fmem{1}{i}$ are
disjoint. Using $\setfn{m}$, we then approximate $\pset{\sets}$ via:
\begin{equation}
	\pset{\sets} \approx
      \bigcup_{\fmem{1}{i}\in \setfn{m}} \pset{\fmem{1}{i}}.
\end{equation}
Using the inclusion-exclusion principle (IEP), we can then approximate $\fxns$
via:
\begin{align}
	\fxn{\pset{\sets}} \approx&
        \sum_{i=1}^m \fxn{\pset{\fmem{1}{i}}} -
        \sum_{i=1}^{m-1}\sum_{j=i+1}^{m}
          \fxn{\pset{\fmem{1}{i}\cap\fmem{1}{j}}} + \cdots +\nonumber\\
      & \left(-1\right)^{m-1}
          \fxn{\pset{\fmem{1}{1}\cap\fmem{1}{2}\cap\cdots\cap\fmem{1}{m}}},
\end{align}
where $\fmem{1}{i}\cap \fmem{1}{j}$ is the intersection of sets $\fmem{1}{i}$
and $\fmem{1}{j}$. For brevity, and because there is a one-to-one mapping
between $\sets$ and its powerset, we define a new function $g$ such that:
\begin{equation}
	\gxn{\sets} \equiv \fxn{\pset{\sets}}.
\end{equation}
In terms of $g$, our first order approximation, $\gapprox{\setfn{m}}$, becomes:
\begin{align}
	\gapprox{\setfn{m}} =&
        \sum_{i=1}^m \gxn{\fmem{1}{i}} -
	    \sum_{i=1}^{m-1}\sum_{j=i+1}^{m}
	      \gxn{\fmem{1}{i}\cap\fmem{1}{j}} + \cdots +\nonumber\\
	  & \left(-1\right)^{m-1}
        \gxn{\fmem{1}{1}\cap\fmem{1}{2}\cap\cdots\cap\fmem{1}{m}}.
\end{align}
To further simplify the equation we write the intersections in terms of the
various $\ints{j}{\setfn{m}}$:
\begin{align}
    \gapprox{\setfn{m}} =&
        \sum_{j=1}^m\sum_{\smem{1,j}{i}\in\ints{j}{\setfn{m}}}
        \left(-1\right)^{j-1}\gxn{\smem{1,j}{i}}.\label{eq:order1}
\end{align}

\newcommand{\dimers}{\mcn{2}{\setfn{m}}}
\newcommand{\lmers}{\mcn{\ell}{\setfn{m}}}

As a second order approximation to $\pset{\sets}$ we take all pairwise unions
of the initial $m$ subsets:
\begin{align}
	\pset{\sets} \approx \bigcup_{\fmem{2}{i}\in\dimers}\pset{\fmem{2}{i}}.
\end{align}
Using the IEP the second order approximation to $\gxn{\sets}$,
$\gapprox{\dimers}$, is:
\begin{align}
	\gapprox{\dimers}=& \sum_{j=1}^{\card{\dimers}}
                        \sum_{\smem{2,j}{i}\in\ints{j}{\dimers}}
                          \left(-1\right)^{j-1}\gxn{\smem{2,j}{i}}.
\end{align}
The above readily generalizes to an $\ell$-order approximation to
$\pset{\sets}$:
\begin{align}
    \pset{\sets} \approx \bigcup_{\fmem{\ell}{i}\in\lmers}\pset{\fmem{l}{i}}.
\end{align}
and the $\ell$-order approximation to $\gxn{\sets}$, $\gapprox{\lmers}$:
\begin{align}
    \gapprox{\lmers}=& \sum_{j=1}^{\card{\lmers}}
                       \sum_{\smem{\ell,j}{i}\in\ints{j}{\lmers}}
                       \left(-1\right)^{j-1}\gxn{\smem{\ell,j}{i}}.
\end{align}
If we flatten the nested families of sets we end up with subsets of $\sets$
which are written as intersections-of-unions of the elements of $\setfn{m}$.
Since intersection distributes over union, it should be possible to rewrite
$\gapprox{\lmers}$ in terms of unions-of-intersections of the elements of
$\setfn{m}$. Inverting this relationship is the first problem we must tackle.

%In practice we actually prefer to write
%$\gapprox{\ell,m}$ as:
%\begin{align}
%\gapprox{\ell,m} = \gapprox{1,m} + \left(\gapprox{2,m} - \gapprox{1,m}\right) +
%                   \cdots +
%                   \left(\gapprox{\ell,m} - \gapprox{\ell - 1,m}\right),
%\end{align}
%which projects $\gapprox{\ell,m}$ into:
%\begin{itemize}
%	\item the first order approximation,
%    \item corrections stemming from the  second order approximation,
%    \item corrections from the third-order through $\left(\ell-1\right)$-order
%          approximations (the ellided terms), and
%    \item the corrections that come from the $\ell$ order approximation.
%\end{itemize}
%In this form, $\gapprox{\ell,m}$ is written in terms of subsets from not only
%$\ints{\ell,m,j}$, but also using subsets from all of the families of sets
%$\ints{k, m, j}$ with $k$ in the range $[1, \ell)$. Writing $\gapprox{\ell,m}$
%in this form, but using the inverted relationships from solving problem one, is
%the second problem.
%
\section{Problem One}

Our overall strategy is deduce the general form of $\gapprox{\lmers}$,
by first determining the general form for select $\ell$ values. Writing the
order-one approximation in terms of the elements of $\setfn{m}$ is
trivial (it is just Eq.~\ref{eq:order1}). The first non-trivial order is
$\ell=2$.

\subsection{$\ell=2$}

\newcommand{\unionset}[2]{\mcn{#1}{\setfn{#2}}}

To deduce the general form for $\ell=2$ we systematically consider increasing
values of $m$. For $m=1$, there is no $\ell=2$ term, and $m=2$ has a trivial
$\ell=2$ term:
\begin{equation}
	\gapprox{\unionset{2}{2}} = \gxn{f_1\cup f_2}.
\end{equation}
Hence the first non-trivial $m$ is $m=3$.


\newcommand{\Di}{f_{12}}
\newcommand{\Dj}{f_{13}}
\newcommand{\Dk}{f_{23}}
\newcommand{\Itwo}[2]{#1\cap #2}
\newcommand{\Ithree}[3]{\Itwo{#1}{#2}\cap #3}
\newcommand{\IDiDj}{\Itwo{\Di}{\Dj}}
\newcommand{\IDiDk}{\Itwo{\Di}{\Dk}}
\newcommand{\IDjDk}{\Itwo{\Dj}{\Dk}}
\newcommand{\IDiDjDk}{\Ithree{\Di}{\Dj}{\Dk}}
\newcommand{\itwo}[2]{\left(#1\cap #2\right)}
\newcommand{\Ii}{s_{12}}
\newcommand{\Ij}{s_{13}}
\newcommand{\Ik}{s_{23}}

For $m=3$ we have:
\begin{align}
	\gapprox{\unionset{2}{3}} =&
      \gxn{\Di} + \gxn{\Dj} + \gxn{\Dk} - \gxn{\IDiDj} -
	                 \gxn{\IDiDk} -\nonumber\\
	               & \gxn{\IDjDk} + \gxn{\IDiDjDk}\label{eq:m2},
\end{align}
where we have further simplified the notation by defining:
\begin{align}
	f_{ijk\cdots} \equiv& f_i\cup f_j\cup f_k \cup \cdots
\end{align}
Noting that:
\begin{equation}
    \left(a\cup b\right) \cap \left(a\cup c\right) = a\cup \left(b\cap c\right)
\end{equation}
and that:
\begin{align}
	\left(a\cup b\right)\cap\left(a\cup c\right)\cap\left(b\cup c\right) =&
	\left(a\cup\left(b\cap c\right)\right)\cap\left(b \cup c\right) \nonumber\\
	=&
	\left(a\cap\left(b\cup c\right)\right)\cup
	\left(\left(b\cap c\right)\cap\left(b\cup c\right)\right)\nonumber\\
		=&
	\left(a\cap b\right)\cup \left(a\cap c\right)\cup \left(b\cap c\right),
\end{align}
we can rewrite Eq.~\eqref{eq:m2} as:
\begin{align}
	\gapprox{\unionset{2}{3}} =& \gapprox{\unionset{2}{2}} + \gxn{\Dj} + \gxn{\Dk} -
	                 \gxn{f_3 \cup \Ii} -
	                 \gxn{f_2 \cup \Ij} -\nonumber\\
	               & \gxn{f_1 \cup \Ik} +
	                 \gxn{\Ii\cup \Ij\cup \Ik}
\end{align}
where we have defined:
\begin{align}
	s_{ijk\cdots} \equiv& f_i\cap f_j\cap f_k \cap \cdots
\end{align}
and identified the $m=2$ approximation.

\newcommand{\Dl}{f_{14}}
\newcommand{\Dm}{f_{24}}
\newcommand{\Dn}{f_{34}}
\newcommand{\Ifour}[4]{\Ithree{#1}{#2}{#3}\cap #4}
\newcommand{\Ifive}[5]{\Ifour{#1}{#2}{#3}{#4}\cap #5}
\newcommand{\Isix}[6]{\Ifive{#1}{#2}{#3}{#4}{#5}\cap #6}
\newcommand{\IDiDl}{\Itwo{\Di}{\Dl}}
\newcommand{\IDiDm}{\Itwo{\Di}{\Dm}}
\newcommand{\IDiDn}{\Itwo{\Di}{\Dn}}
\newcommand{\IDjDl}{\Itwo{\Dj}{\Dl}}
\newcommand{\IDjDm}{\Itwo{\Dj}{\Dm}}
\newcommand{\IDjDn}{\Itwo{\Dj}{\Dn}}
\newcommand{\IDkDl}{\Itwo{\Dk}{\Dl}}
\newcommand{\IDkDm}{\Itwo{\Dk}{\Dm}}
\newcommand{\IDkDn}{\Itwo{\Dk}{\Dn}}
\newcommand{\IDlDm}{\Itwo{\Dl}{\Dm}}
\newcommand{\IDlDn}{\Itwo{\Dl}{\Dn}}
\newcommand{\IDmDn}{\Itwo{\Dm}{\Dn}}
\newcommand{\IDiDjDl}{\Ithree{\Di}{\Dj}{\Dl}}
\newcommand{\IDiDjDm}{\Ithree{\Di}{\Dj}{\Dm}}
\newcommand{\IDiDjDn}{\Ithree{\Di}{\Dj}{\Dn}}
\newcommand{\IDiDkDl}{\Ithree{\Di}{\Dk}{\Dl}}
\newcommand{\IDiDkDm}{\Ithree{\Di}{\Dk}{\Dm}}
\newcommand{\IDiDkDn}{\Ithree{\Di}{\Dk}{\Dn}}
\newcommand{\IDiDlDm}{\Ithree{\Di}{\Dl}{\Dm}}
\newcommand{\IDiDlDn}{\Ithree{\Di}{\Dl}{\Dn}}
\newcommand{\IDiDmDn}{\Ithree{\Di}{\Dm}{\Dn}}
\newcommand{\IDjDkDl}{\Ithree{\Dj}{\Dk}{\Dl}}
\newcommand{\IDjDkDm}{\Ithree{\Dj}{\Dk}{\Dm}}
\newcommand{\IDjDkDn}{\Ithree{\Dj}{\Dk}{\Dn}}
\newcommand{\IDjDlDm}{\Ithree{\Dj}{\Dl}{\Dm}}
\newcommand{\IDjDlDn}{\Ithree{\Dj}{\Dl}{\Dn}}
\newcommand{\IDjDmDn}{\Ithree{\Dj}{\Dm}{\Dn}}
\newcommand{\IDkDlDm}{\Ithree{\Dk}{\Dl}{\Dm}}
\newcommand{\IDkDlDn}{\Ithree{\Dk}{\Dl}{\Dn}}
\newcommand{\IDkDmDn}{\Ithree{\Dk}{\Dm}{\Dn}}
\newcommand{\IDlDmDn}{\Ithree{\Dl}{\Dm}{\Dn}}
\newcommand{\IDiDjDkDl}{\Ifour{\Di}{\Dj}{\Dk}{\Dl}}
\newcommand{\IDiDjDkDm}{\Ifour{\Di}{\Dj}{\Dk}{\Dm}}
\newcommand{\IDiDjDkDn}{\Ifour{\Di}{\Dj}{\Dk}{\Dn}}
\newcommand{\IDiDjDlDm}{\Ifour{\Di}{\Dj}{\Dl}{\Dm}}
\newcommand{\IDiDjDlDn}{\Ifour{\Di}{\Dj}{\Dl}{\Dn}}
\newcommand{\IDiDjDmDn}{\Ifour{\Di}{\Dj}{\Dm}{\Dn}}
\newcommand{\IDiDkDlDm}{\Ifour{\Di}{\Dk}{\Dl}{\Dm}}
\newcommand{\IDiDkDlDn}{\Ifour{\Di}{\Dk}{\Dl}{\Dn}}
\newcommand{\IDiDkDmDn}{\Ifour{\Di}{\Dk}{\Dm}{\Dn}}
\newcommand{\IDiDlDmDn}{\Ifour{\Di}{\Dl}{\Dm}{\Dn}}
\newcommand{\IDjDkDlDm}{\Ifour{\Dj}{\Dk}{\Dl}{\Dm}}
\newcommand{\IDjDkDlDn}{\Ifour{\Dj}{\Dk}{\Dl}{\Dn}}
\newcommand{\IDjDkDmDn}{\Ifour{\Dj}{\Dk}{\Dm}{\Dn}}
\newcommand{\IDjDlDmDn}{\Ifour{\Dj}{\Dl}{\Dm}{\Dn}}
\newcommand{\IDkDlDmDn}{\Ifour{\Dk}{\Dl}{\Dm}{\Dn}}
\newcommand{\IDiDjDkDlDm}{\Ifive{\Di}{\Dj}{\Dk}{\Dl}{\Dm}}
\newcommand{\IDiDjDkDlDn}{\Ifive{\Di}{\Dj}{\Dk}{\Dl}{\Dn}}
\newcommand{\IDiDjDkDmDn}{\Ifive{\Di}{\Dj}{\Dk}{\Dm}{\Dn}}
\newcommand{\IDiDjDlDmDn}{\Ifive{\Di}{\Dj}{\Dl}{\Dm}{\Dn}}
\newcommand{\IDiDkDlDmDn}{\Ifive{\Di}{\Dk}{\Dl}{\Dm}{\Dn}}
\newcommand{\IDjDkDlDmDn}{\Ifive{\Dj}{\Dk}{\Dl}{\Dm}{\Dn}}
\newcommand{\IDiDjDkDlDmDn}{\Isix{\Di}{\Dj}{\Dk}{\Dl}{\Dm}{\Dn}}
\newcommand{\Il}{s_{14}}
\newcommand{\In}{s_{24}}
\newcommand{\Io}{s_{34}}
\newcommand{\Iii}{s_{123}}
\newcommand{\Ijj}{s_{124}}
\newcommand{\Ikk}{s_{134}}
\newcommand{\Ill}{s_{234}}

\newcommand{\intsset}[3]{\ints{#1}{\unionset{#2}{#3}}}

For $m=4$ there are 63 terms. Stemming from the $\intsset{1}{2}{4}$ family of
sets are 6 trivial terms:
\begin{align}
	\sum_{\fmem{2}{i}\in\intsset{1}{2}{4}}\gxn{\fmem{2}{i}} =&
	  \gxn{\Di} + \gxn{\Dj} + \gxn{\Dk} + \gxn{\Dl} +
	  \gxn{\Dm} + \gxn{\Dn}\nonumber\\
	=& \sum_{\fmem{2}{i}\in\intsset{1}{2}{3}}\gxn{\fmem{2}{i}} +
	   \sum_{\fmem{2}{i}\in\intsset{1}{2}{4}\setminus\intsset{1}{2}{3}}
	    \gxn{\fn{2}_i},
\end{align}
where in the second line we separated the six terms into the three that are
from the $\intsset{1}{2}{3}$ family of sets and the three which are unique to
the $\intsset{1}{2}{4}$ family of sets. This is straightforward to generalize
to arbitrary $m$:
\begin{align}
    \sum_{\fmem{2}{i}\in\intsset{1}{2}{m}}\gxn{\fmem{2}{i}} =
    \sum_{\fmem{2}{i}\in\intsset{1}{2}{m-1}}\gxn{\fmem{2}{i}} +
    \sum_{\fmem{2}{i}\in\intsset{1}{2}{m}\setminus\intsset{1}{2}{m-1}}
    \gxn{\fn{2}_i}.
\end{align}

Stemming from $\intsset{2}{2}{4}$ are 15 terms:
\begin{align}
	\sum_{\smem{2,2}{i}\in\intsset{2}{2}{4}}\gxn{\smem{2,2}{i}} =
	  & \gxn{\IDiDj} + \gxn{\IDiDk} + \gxn{\IDiDl} + \gxn{\IDiDm} +\nonumber\\
	  & \gxn{\IDiDn} + \gxn{\IDjDk} + \gxn{\IDjDl} + \gxn{\IDjDm} +\nonumber\\
	  & \gxn{\IDjDn} + \gxn{\IDkDl} + \gxn{\IDkDm} + \gxn{\IDkDn} +\nonumber\\
	  & \gxn{\IDlDm} + \gxn{\IDlDn} + \gxn{\IDmDn},
\end{align}
which respectively simplify to:
\begin{align}
	\sum_{\smem{2,2}{i}\in\intsset{2}{2}{4}}\gxn{\smem{2,2}{i}} =
	& \gxn{f_1\cup\Ik} + \gxn{f_2\cup\Ij} + \gxn{f_1\cup\In} +
	  \gxn{f_2\cup\Il} + \nonumber\\
	& \gxn{\Ij\cup\Ik\cup\Il\cup\In} + \gxn{f_3\cup\Ii} +
	  \gxn{f_1\cup\Io} + \nonumber\\
	& \gxn{\Ii\cup\Ik\cup\Il\cup\Io} + \gxn{f_3\cup\Il}
	+ \gxn{\Ii\cup\Ij\cup\In\cup\Io} + \nonumber\\
	& \gxn{f_2\cup\Io} + \gxn{f_3\cup\In} + \gxn{f_4\cup\Ii} +
	  \gxn{f_4\cup\Ij} + \gxn{f_4\cup\Ik},
\end{align}
where we have used:
\begin{align}
	\left(a\cup b\right)\cap\left(c\cup d\right) =&
	  \left(a\cap\left(c\cup d\right)\right)\cup
	  \left(b\cap\left(c\cup d\right)\right)\nonumber\\
	   =& \left(a\cap c\right)\cup\left(a\cap d\right)\cup
          \left(b\cap c\right)\cup\left(b\cap d\right)
\end{align}
Identifying the terms which were present for $\intsset{2}{2}{3}$:
\begin{align}
	\sum_{\smem{2,2}{i}\in\intsset{2}{2}{4}}\gxn{\smem{2,2}{i}} =
	& \sum_{\smem{2,2}{i}\in\intsset{2}{2}{3}}\gxn{\smem{2,2}{i}} +
      \gxn{f_1\cup\In} + \gxn{f_2\cup\Il} + \nonumber\\
	& \gxn{\Ij\cup\Ik\cup\Il\cup\In}  +  \gxn{f_1\cup\Io} + \nonumber\\
	& \gxn{\Ii\cup\Ik\cup\Il\cup\Io} + \gxn{f_3\cup\Il}
	+ \gxn{\Ii\cup\Ij\cup\In\cup\Io} + \nonumber\\
	& \gxn{f_2\cup\Io} + \gxn{f_3\cup\In} + \gxn{f_4\cup\Ii} +
	\gxn{f_4\cup\Ij} + \gxn{f_4\cup\Ik}.
\end{align}


Stemming from $\intsset{3}{2}{4}$ are 20 terms:
\begin{align}
	\sum_{\smem{2,3}{i}\in\intsset{3}{2}{4}}\gxn{\smem{2,3}{i}} =
	  & \gxn{\IDiDjDk} + \gxn{\IDiDjDl} + \gxn{\IDiDjDm} + \nonumber\\
	  & \gxn{\IDiDjDn} + \gxn{\IDiDkDl} + \gxn{\IDiDkDm} + \nonumber\\
	  & \gxn{\IDiDkDn} + \gxn{\IDiDlDm} + \gxn{\IDiDlDn} + \nonumber\\
	  & \gxn{\IDiDmDn} + \gxn{\IDjDkDl} + \gxn{\IDjDkDm} + \nonumber\\
	  & \gxn{\IDjDkDn} + \gxn{\IDjDlDm} + \gxn{\IDjDlDn} + \nonumber\\
	  & \gxn{\IDjDmDn} + \gxn{\IDkDlDm} + \gxn{\IDkDlDn} + \nonumber\\
	  & \gxn{\IDkDmDn} + \gxn{\IDlDmDn},
\end{align}
which respectively simplify to:
\begin{align}
	\sum_{\smem{2,3}{i}\in\intsset{3}{2}{4}}\gxn{\smem{2,3}{i}} =
	& \gxn{\Ii\cup\Ij\cup\Ik} + \gxn{f_1\cup\Ill} +
	  \gxn{\Ii\cup\Il\cup\Ik} + \nonumber\\
	& \gxn{\Ij\cup\Il\cup\Ik} + \gxn{\Ii\cup\Ij\cup\In} +
	  \gxn{f_2\cup\Ikk} + \nonumber\\
	& \gxn{\Ij\cup\Ik\cup\In} + \gxn{\Ii\cup\Il\cup\In} +
	  \gxn{\Ij\cup\Il\cup\In} + \nonumber\\
	& \gxn{\Il\cup\Ik\cup\In} + \gxn{\Ii\cup\Ij\cup\Io} +
	  \gxn{\Ii\cup\Ik\cup\Io} + \nonumber\\
	& \gxn{f_3\cup\Ijj} + \gxn{\Ii\cup\Il\cup\Io} +
	  \gxn{\Ij\cup\Il\cup\Io} + \nonumber\\
	& \gxn{\Il\cup\Ik\cup\Io} + \gxn{\Ii\cup\In\cup\Io} +
	  \gxn{\Ij\cup\In\cup\Io} +\nonumber\\
	& \gxn{\Ik\cup\In\cup\Io} + \gxn{f_4\cup\Iii}.
\end{align}
Identifying the terms which stem from $\intsset{2}{3}{3}$:
\begin{align}
	\sum_{\smem{2,3}{i}\in\intsset{2}{3}{4}}\gxn{\smem{2,3}{i}} =
	& \sum_{\smem{2,3}{i}\in\intsset{2}{3}{3}}\gxn{\smem{2,3}{i}} +
	  \gxn{f_1\cup\Ill} + \gxn{\Ii\cup\Il\cup\Ik} + \nonumber\\
	& \gxn{\Ij\cup\Il\cup\Ik} + \gxn{\Ii\cup\Ij\cup\In} +
	\gxn{f_2\cup\Ikk} + \nonumber\\
	& \gxn{\Ij\cup\Ik\cup\In} + \gxn{\Ii\cup\Il\cup\In} +
	\gxn{\Ij\cup\Il\cup\In} + \nonumber\\
	& \gxn{\Il\cup\Ik\cup\In} + \gxn{\Ii\cup\Ij\cup\Io} +
	\gxn{\Ii\cup\Ik\cup\Io} + \nonumber\\
	& \gxn{f_3\cup\Ijj} + \gxn{\Ii\cup\Il\cup\Io} +
	\gxn{\Ij\cup\Il\cup\Io} + \nonumber\\
	& \gxn{\Il\cup\Ik\cup\Io} + \gxn{\Ii\cup\In\cup\Io} +
	\gxn{\Ij\cup\In\cup\Io} +\nonumber\\
	& \gxn{\Ik\cup\In\cup\Io} + \gxn{f_4\cup\Iii}.
\end{align}
At this point we have identified all terms which are also present in
$\gapprox{\unionset{2}{3}}$ and remaining terms are unique to
$\gapprox{\unionset{2}{4}}$.


Stemming from $\intsset{4}{2}{4}$ are 15 terms:
\begin{align}
	\sum_{\smem{2,4}{i}\in\intsset{4}{2}{4}}\gxn{\smem{2,4}{i}} =
	  & \gxn{\IDiDjDkDl} + \gxn{\IDiDjDkDm} + \nonumber\\
	  & \gxn{\IDiDjDkDn} + \gxn{\IDiDjDlDm} + \nonumber\\
	  & \gxn{\IDiDjDlDn} + \gxn{\IDiDjDmDn} + \nonumber\\
	  & \gxn{\IDiDkDlDm} + \gxn{\IDiDkDlDn} + \nonumber\\
	  & \gxn{\IDiDkDmDn} + \gxn{\IDiDlDmDn} + \nonumber\\
	  & \gxn{\IDjDkDlDm} + \gxn{\IDjDkDlDn} + \nonumber\\
	  & \gxn{\IDjDkDmDn} + \gxn{\IDjDlDmDn} + \nonumber\\
	  & \gxn{\IDkDlDmDn},
\end{align}
which respectively simplify to:

\begin{align}
	\sum_{\smem{2,4}{i}\in\intsset{4}{2}{4}}\gxn{\smem{2,4}{i}} =
	& \gxn{\Ii\cup\Ij\cup\Ill} + \gxn{\Ii\cup\Ik\cup\Ikk} +
	  \gxn{\Ij\cup\Ik\cup\Ijj} + \nonumber\\
	& \gxn{\Ii\cup\Il\cup\Ill} + \gxn{\Ij\cup\Il\cup\Ill} +
	  \gxn{\Il\cup\Ik} + \nonumber\\
	& \gxn{\Ii\cup\In\cup\Ikk} + \gxn{\Ij\cup\In} +
	  \gxn{\Ik\cup\In\cup\Ikk} + \nonumber\\
	& \gxn{\Il\cup\In\cup\Iii} + \gxn{\Ii\cup\Io} +
	  \gxn{\Ij\cup\Io\cup\Ijj} + \nonumber\\
	& \gxn{\Ik\cup\Io\cup\Ijj} + \gxn{\Il\cup\Io\cup\Iii} +
	  \gxn{\In\cup\Io\cup\Iii},
\end{align}

Stemming from $\intsset{5}{2}{4}$ are 6 terms:
\begin{align}
	\sum_{\smem{2,5}{i}\in\intsset{5}{2}{4}}\gxn{\smem{2,5}{i}} =
 	  & \gxn{\IDiDjDkDlDm} + \gxn{\IDiDjDkDlDn} + \nonumber\\
	  & \gxn{\IDiDjDkDmDn} + \gxn{\IDiDjDlDmDn} + \nonumber\\
	  & \gxn{\IDiDkDlDmDn} + \gxn{\IDjDkDlDmDn},
\end{align}
which respectively simplify to:

\begin{align}
	\sum_{\smem{2,5}{i}\in\intsset{5}{2}{4}}\gxn{\smem{2,5}{i}} =
	& \gxn{\Ii\cup\Ikk\cup\Ill} + \gxn{\Ij\cup\Ijj\cup\Ill} +
	  \gxn{\Ik\cup\Ijj\cup\Ikk} + \nonumber\\
	& \gxn{\Il\cup\Iii\cup\Ill} + \gxn{\In\cup\Iii\cup\Ikk} +
	  \gxn{\Io\cup\Iii\cup\Ijj},
\end{align}

Finally there is one term stemming from $\intsset{6}{2}{4}$:
\begin{align}
    \sum_{\smem{2,6}{i}\in\intsset{6}{2}{4}}\gxn{\smem{2,6}{i}} =
      \gxn{\IDiDjDkDlDmDn} = \gxn{\Iii\cup\Ijj\cup\Ikk\cup\Ill}
\end{align}


At this point we make a number of observations:
\begin{itemize}
	\item There are $2^{m \choose 2} - 1$ terms. So $m=5$ and $6$
	      respectively contain 1023, and 32,768 terms. Meaning explicitly
	      writing out higher-orders is impractical.
	\item The expressions involve unions of elements from $\unionset{2}{m}$,
          $\unionset{1}{m}$, and  $\iset{\unionset{1}{m}}$ instead of
          intersections of elements from $\iset{\unionset{2}{m}}$.
	\item Algorithmically this allows us to reuse information from the
	      dramatically smaller $\iset{\unionset{1}{m}}$ family of sets (namely
          which elements of $\iset{\unionset{1}{m}}$ are empty) to decrease the
          number of terms we must consider.
\end{itemize}


\end{document}
